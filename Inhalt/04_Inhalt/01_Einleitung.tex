
\chapter{Einleitung}
Während dieser Hausarbeit wird der Qualitätssicherungsprozess des \ac{TM}-\ac{CoE}s an dem Beispiel des \ac{TM}-Migrationsservices betrachtet.\\
Da das \ac{CoE} als Abteilung Teil des Kundensupports ist, wird er Qualitätssicherungsprozess mit Teilen der ISO 9001 analysiert.
Zunächst beschreibe ich die Abteilung in der nächsten Sektion
\\
\section{Abteilung}
Das \ac{CoE} fokussiert sich auf das schnelle Lösen von Kundenproblemen und ist somit Teil des Kundensupports.
Hierfür bietet das \ac{CoE} mehrere Services an, welche ein Kunden buchen kann, um Probleme zu lösen.
\\
Das \ac{TM}-\ac{CoE} ist Teil des Logistik-\ac{CoE} und hilft Kunden bei der Transportlogistikstoftware SAP \ac{TM}.
Eine Teil des Angebots des \ac{TM}-\ac{CoE} ist der Migrationsservice, welcher inder der nächsten Sektion beschreiben wird.
\section{Migrationsservice}
Beim Migrationsservice hilft das \ac{TM}-\ac{CoE} bei der Übertragung der Daten von den alleinstehenden \ac{TM}-System auf ein eingebundenes S/4HANA-System.
Hierbei fokussiert sich der Service auf die Übertragung von Einstellung und Stammdaten.
Die Bewegungsdaten werden von einem anderen SAP-Team übertragen.
\\
Während dieses Services werden drei Testmigrationen mit aufsteigender Komplexität durchgeführt.
Nach diesen drei Testmigrationen findet an einem Wochenende eine Produktivmigration statt. 
\\
Dieser Service wurde bisher einmal vollständig geliefert.
Diese Lieferung war beim Kunden \ac{VW}.
\\
Während dieses Services gibt es einen Kundenansprechpartner, welchen SAP \ac{TQM} nennt, welcher für dieses Engagement verantwortlich ist.
{\let\clearpage\relax \chapter{PDCA-Kreis}}
Wie funktioniert jedoch der Qualitätssicherungsprozess für den Migrationsservice.
Hierzu kann der PDCA-Zyklus der ISO 9001 betrachtet werden.
Da die ISO 9001 jedoch für ganze Unternehmen entwickelt wurde, kann nur ein Teil dieser Norm für das \ac{CoE} und den Migrationsservice betrachtet werden. 
\\
Der PDCAZyklus ist in die Teile Plan, Do, Check und Act unterteilt.
\newline
\newline
Der Plan-Teil ist hierbei nochmal in die Teile Kontext der Organisation, Führung und Planung unterteilt.
Bei Kontext der Organisation wird betrachtet, wie das Qualitätsmanagement im Unternehmen umsetzbar ist und woher die Anforderungen für den Prozess kommen.
Der Teil Führung fordert die ISO 9001, dass das Qualitätsmanagement von oberster Führungsebene kommt und betrachtet wird.
Der Planungsteil betrachtet, wie der Prozess umgesetzt werden soll. 
Hierbei soll besonders auf Anforderungen, Risiken und Vorgehen eingegangen werden.
\newline
\newline
Wie ist die Planung im \ac{TM}-\ac*{CoE} umgesetzt.
Hierzu wird zunächst von einem globalen \ac{CoE} der Service vorgegeben.
Bei dem Migrationsservice war dies eine Migration von alten Netweaver-Systemen auf neue S/4HANA-Systeme.
\\
Da bei \ac{TM} jedoch der Migrationsservice nicht in dem Umfang, welcher vom globalen \ac{CoE} vorgegeben wurde, umgesetzt werden kann, da nur Customizing, Stammdaten und Einstellungen übertragen werden können, muss das \ac{TM}-\ac{CoE} diese Anforderungen anpassen.
Diese Anforderungsanpassung nimmt der Manager vor. Diese Anforderungen werden dann mit dem Team über eine Sharepointseite, welche über das \ac{MS}-Teams team zufinden ist.
\\
Mit diesen Anforderungen wird dannach ein oder mehrere Teammitglieder damit beauftragt den Serviceinhalt auszuarbeiten.
Hierbei werden sowohl die initialen Vorstellungsslides für den Verkaufs-/Einschätzungsworkshop als auch Vorlagen für den realen Service.
\\
Beim Migrationsservice wurde dieser Servicecontent erstellt, da der Fokus des Services jedoch mehr auf der Migration lag musste das Migrationstool der Entwicklungsabteilung genutzt werden.
Da dieses Tool noch nicht genutzt wurde, diente der Migrationsservice auch dazu das Tool zu testen und Empfehlungen zur Verbesserung vorzuschlagen.
\newline
\newline
Als nächstes bewertet das \ac{CoE} die Dauer dieses Services.
Hierzu wird als Einheit, welche verkauft wird Manntage genutzt.
Diese Schätzung des Services wird vom Manager oder von dem Teamlead durchgeführt.
Eine Grundschätztung für die Anzahl der Manntage wird vom Globale \ac*{CoE} vorgegeben. 
